\documentclass[11pt,letterpaper]{article}
\usepackage{tocloft}
\usepackage[margin=0.8in]{geometry}
\begin{document}

\subsection*{The Boys of Ballycastle / Off to California / The Boys of Bluehill}

These are some of the first hornpipes I learned early on. The Boys of Ballycastle, I think I learned from Kevin Burke's album "Up Close"

Off to California and The Boys of Bluehill, I most likely learned from sessions around Connecticut. Sessions hosted by folks like P.V. O'Donnel, Eddie Burke, John Kalinowski, and others.

\subsection*{A Polka / A Finnish Polka / Jessica's Polka}

This set I took straight from Kevin Burke's "Up Close"

\subsection*{The Blarney Pilgrim / Garret Barry's / Banish Misfortune}

I believe that I learned Blarney Pilgrim and Garret Barry's from a book that a friend had. Banish Misfortune is one that I picked up from those early sessions.

\subsection*{The Yellow Tinker / Doonagore / The Fermoy Lasses}

The Yellow Tinker I learned at the Wednesday night session at Riley's Pub in St Louis. Jerry Maloney brought that tune into the session.

Doonagore I learned from a recording of Michael O'Raghallaigh. The Fermoy Lasses, I either learned at those early sessions in New Orleans or In Connecticut, or I pulled it out of a book. I know it's one I've known for a long time. It shows up in a couple different places in the tunebook because I was always trying to find a home for it that felt right. 

\subsection*{The Mason's Apron / Tam Lin / Master Crowley's}

The Mason's Apron and Tam Lin I learned from a mandolin player I stumbled across on YouTube, Dan Beimborn. He played those two in a set followed by a tune he wrote called The Banjo Reel. I never learned the one he wrote, but when I learned Master Crowley's from a recording of Sean Keane and his son Padraig, I felt that if fit nicely after Tam Lin.

\subsection*{Julia Delaney / Ships are Sailing / The Star of Munster}
Julia Delaney is one that I know I learned while living in Connecticut. Ships are Sailing I learned a few years ago after moving to West Lafayette and starting the session here. That's one that I had been hearing and sought out a few recordings to learn it from. Mostly people on YouTube.

The Star of Munster I think has only been in my repertoire for a similar amount of time. But I don't remember where or how I learned it. 

\subsection*{Out on the Ocean / The Shandon Bells / The Connaughtman's Rambles}

More tunes that have been in my repertoire from pretty early on. Out on the Ocean and Connaughtman's Rambles are super common at sessions. The Shandon Bells, I learned from an album by the Greenfield Dance Band. I really like this tune, but I almost never come across other people who know it. 

\subsection*{The Road to Lisdoonvarna / The Swallowtail Jig / The Fermoy Lasses}

More tunes I've known forever. I put this set together for the novelty of having three types of tunes (a slide, a jig, and a reel) that are all very similar but increasing in "notes per measure".

\subsection*{The Road to Lisdoonvarna / The Swallowtail Jig / The Kesh Jig}

This started out as the previous set, but I was preparing to play a St. Patrick's Day gig with Chops MacConnie. I presented this set. He didn't know The Fermoy Lasses, and suggested putting The Kesh in its place. This has since become a regular set at the West Lafayette session.

\subsection*{The Frost is All Over / Kitty Lie Over / Cunla}

All three of these are jigs that are songs as well. If I'm remembering correctly I learned all of them from the singing of Johnny Moynihan. What I'm calling The Frost is All Over and Kitty Lie Over are two different melodies that the same lyrics are sung to. Famously Johnny Moynihan sang these together with Planxty. 

The song Cunla, as I learned several years later, is sung to the first three parts of the five part jig The Freize Britches.

\subsection*{Where I Told Her I Loved Her and Sprained Her Ankle / Gander in the Pratie Hole / The Persistence of Noel Reid}

The first and third of these are my own compositions. The first I named for the place where I first said "I love you" to the woman I would later marry. It was a pier made of granite blocks in Stonington Point in Connecticut. On the walk back in to firm ground she slipped and sprained her ankle. 

The Persistence of Noel Reid is named for the night I was introduced to Irish set dancing and the tunes they are danced to. I was living in New Orleans at the time. This would be about 2003. I had fallen in with a group of shanty singers, The NO Quarter Shanty Krewe. Some of them also played Irish music, and one night they invited to a ceili. I didn't know what a ceili was. I was told that there would be live music and dancing. We would get to sing a little, and people would be generally quite nice. 
I arrived where I was told go but was confused upon my arrival as I looked around and say people playing pool, country music on the juke box and a lot of leather jackets by the bar. I was pretty sure that there was only one Mick's Pub in Mid City. I was then relieved to see stairs in the corner. I went up the stairs to find a room with at least a dozen musicians lined up along two walls, just enough room for two sets of dancers, and everyone else pressed up against the walls, carrying on conversations while trying to stay out of the way. 

Confident that I was, in fact, in the right place, I went back to the bar, grabbed a drink, and re-entered the room. At this point they dance they were doing had just finished, and a man named Noel was yelling "We need more men for this set". Keep in mind that at this point I was about 25 years old. I had never danced before. I did not think that I was interested in starting to dance. But he urged me toward the floor, I thought "why not" and set down my drink. Noel placed me in front of someone, said "This is Amy. She knows what she's doing." And walked away. 

I ended up having a blast the rest of the night. And ever since I have been hooked on Irish traditional music, and set dancing. I attended the weekly set dancing lesson, the monthly ceilis, and the regular sessions for that last year that I lived in New Orleans. 

Gander in the Pratie Hole I first tried to learn from a book I was given. I think I actually learned it later from a Planxty album. 

\subsection*{Where I Told Her I Loved Her and Sprained Her Ankle / Coffee on the Bricks / The Persistence of Noel Reid}

This set is simply the previous set, except that I replaced Gander in the Pratie Hole with Coffee on the Bricks, which is another tune of my own composition. This would make for the first time I was able to put together a set of three tunes all of which were mine. 

Coffee on the Bricks is named for both the brick paved historic Main St in St Charles, Missouri, and Picasso's Coffee House on that street. 

During the time we lived in St Charles, I was staying home with the kids. We only lived a couple of blocks from Main St so I frequently put the kids in a stroller and walked them down to Main St, and Frontier Park on the Missouri River. This also usually involved a stop at Picasso's, where much of my social life revolved around at the time. 

As the kids got a little older we would make the walk down to Main St without the stroller, and I would often buy the kids hot chocolate as well. My youngest, Ren, would refer these walks as "getting coffee on the bricks".

\subsection*{The Walls of Liscarroll / Behind the Haystack / Merrily Kissed the Quaker}

At the moment I'm having trouble remembering where I learned each of these specific tunes. I know that I put them together because I thought they worked well together and made nice transitions from one to the next. 

\subsection*{Tar Road to Sligo / The Cliffs of Moher / Salt River Road}

The first two I learned as set from one of John Whelan's albums. Salt River Road is my composition. I named it for a road in St Charles, Missouri. I remember it as being one of the first exit signs I would see on the highway after leaving our house. 

\subsection*{Maggie in the Woods / Peggy Lettermore}

These I learned from the regular Wednesday night sessions at Riley's Pub in St Louis. I believe Jerry Maloney put these together as a set. 

\subsection*{Lilting Banshee / Coffee / Tripping up the Stairs}

I'm not sure where specifically I learned Lilting Banshee, or Tripping up the Stairs. I know that they were in my repertoire pretty early on. Coffee I know was popular among the contra dance players I was around a lot in Connecticut. And I think we all learned it from one of The Portland Collection books. \textbf{(I need to look up who the composer is)}

\subsection*{Dinky's / The Ash Plant / Return to Milltown}

I don't remember where I learned Dinky's. It might have been from hearing it played by The Great Bear Trio, and then looking it up later. The Ash Plant I learned from the Noel Hill and Tony McMahon album I gCnoc na Graí. Return to Milltown I learned from a Greenfield Dance Band album. \textbf{need to look up composer}

\end{document}

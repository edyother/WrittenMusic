\documentclass[11pt,letterpaper]{article}
\usepackage{tocloft}
\usepackage[margin=0.8in]{geometry}
\begin{document}

\subsection*{The Boys of Ballycastle / Off to California / The Boys of Bluehill}

These are some of the first hornpipes I learned early on. The Boys of Ballycastle, I think I learned from Kevin Burke's album "Up Close"

Off to California and The Boys of Bluehill, I most likely learned from sessions around Connecticut. Sessions hosted by folks like P.V. O'Donnel, Eddie Burke, John Kalinowski, and others.

\subsection*{A Polka / A Finnish Polka / Jessica's Polka}

This set I took straight from Kevin Burke's "Up Close"

\subsection*{The Blarney Pilgrim / Garret Barry's / Banish Misfortune}

I believe that I learned Blarney Pilgrim and Garret Barry's from a book that a friend had. Banish Misfortune is one that I picked up from those early sessions.

\subsection*{The Yellow Tinker / Doonagore / The Fermoy Lasses}

The Yellow Tinker I learned at the Wednesday night session at Riley's Pub in St Louis. Jerry Maloney brought that tune into the session.

Doonagore I learned from a recording of Micheal O'Riley (look up the spelling of his name). The Fermoy Lasses, I either learned at those early sessions in New Orleans or In Connecticut, or I pulled it out of a book. I know it's one I've known for a long time. It shows up in a couple different places in the tunebook because I was always trying to find a home for it that felt right. 

\subsection*{The Mason's Apron / Tam Lin / Master Crowley's}

The Mason's Apron and Tam Lin I learned from a mandolin player I stumbled across on YouTube, Dan Beimborn (check spelling). He played those two in a set followed by a tune he wrote called The Banjo Reel. I never learned the one he wrote, but when I learned Master Crowley's from a recording of Sean Keane and his son Padraig, I felt that if fit nicely after Tam Lin.

\subsection*{Julia Delaney / Ships are Sailing / The Star of Munster}
Julia Delaney is one that I know I learned while living in Connecticut. Ships are Sailing I learned a few years ago after moving to West Lafayette and starting the session here. That's one that I had been hearing and sought out a few recordings to learn it from. Mostly people on YouTube.

The Star of Munster I think has only been in my repertoire for a similar amount of time. But I don't remember where or how I learned it. 

\subsection*{Out on the Ocean / The Shandon Bells / The Connaughtman's Rambles}

More tunes that have been in my repertoire from pretty early on. Out on the Ocean and Connaughtman's Rambles are super common at sessions. The Shandon Bells, I learned from an album by the Greenfield Dance Band. I really like this tune, but I almost never come across other people who know it. 

\subsection*{The Road to Lisdoonvarna / The Swallowtail Jig / The Fermoy Lasses}

More tunes I've known forever. I put this set together for the novelty of having three types of tunes (a slide, a jig, and a reel) that are all very similar but increasing in "notes per measure".

\subsection*{The Road to Lisdoonvarna / The Swallowtail Jig / The Kesh Jig}

This started out as the previous set, but I was preparing to play a St. Patrick's Day gig with Chops MacConnie. I presented this set. He didn't know The Fermoy Lasses, and suggested putting The Kesh in its place. This has since become a regular set at the West Lafayette session.





\end{document}

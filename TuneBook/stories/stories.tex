\documentclass[11pt,letterpaper]{article}
\usepackage{tocloft}
\usepackage[margin=0.8in]{geometry}
\usepackage[bottom]{footmisc}
\usepackage{indentfirst}
\title{Behind the Sets}
\author{Ed Yother\\ed@edyother.com\\edyother.com}
\begin{document}

\renewcommand{\thesection}{}
\renewcommand{\thesubsection}{}
\pagenumbering{roman}
\maketitle

\tableofcontents

\newpage

\pagenumbering{arabic}

\section{The Sets and Thier Stories}

\subsection{The Boys of Ballycastle / Off to California / The Boys of Bluehill}

These are some of the first hornpipes I learned early on. The Boys of Ballycastle, I think I learned from Kevin Burke's album "Up Close"

Off to California and The Boys of Bluehill, I most likely learned from sessions around Connecticut. Sessions hosted by folks like P.V. O'Donnel, Eddie Burke, Tom Walsh, John Kalinowski, John Whelan, Jeanne Freeman, and others.

\subsection{A Polka / A Finnish Polka / Jessica's Polka}

This set I took straight from Kevin Burke's "Up Close"

\subsection{The Blarney Pilgrim / Garret Barry's / Banish Misfortune}

I believe that I learned Blarney Pilgrim and Garret Barry's from a book that a friend had. Banish Misfortune is one that I picked up from those early sessions.

\subsection{The Yellow Tinker / Doonagore / The Fermoy Lasses}

The Yellow Tinker I learned at the Wednesday night session at Riley's Pub in St Louis. Jerry Maloney brought that tune into the session.

Doonagore I learned from a recording of Michael O'Raghallaigh. The Fermoy Lasses, I either learned at those early sessions in New Orleans or In Connecticut, or I pulled it out of a book. I know it's one I've known for a long time. It shows up in a couple different places in the tunebook because I was always trying to find a home for it that felt right. 

\subsection{The Mason's Apron / Tam Lin / Master Crowley's}

The Mason's Apron and Tam Lin I learned from a mandolin player I stumbled across on YouTube, Dan Beimborn. He played those two in a set followed by a tune he wrote called The Banjo Reel. I never learned the one he wrote, but when I learned Master Crowley's from a recording of Sean Keane and his son Padraig, I felt that if fit nicely after Tam Lin.

\subsection{Julia Delaney / Ships are Sailing / The Star of Munster}
Julia Delaney is one that I know I learned while living in Connecticut. Ships are Sailing I learned a few years ago after moving to West Lafayette and starting the session here. That's one that I had been hearing and sought out a few recordings to learn it from. Mostly people on YouTube.

The Star of Munster I think has only been in my repertoire for a similar amount of time. But I don't remember where or how I learned it. 

\subsection{Out on the Ocean / The Shandon Bells / The Connaughtman's Rambles}

More tunes that have been in my repertoire from pretty early on. Out on the Ocean and Connaughtman's Rambles are super common at sessions. The Shandon Bells, I learned from an album by the Greenfield Dance Band. I really like this tune, but I almost never come across other people who know it. 

\subsection{The Road to Lisdoonvarna / The Swallowtail Jig / The Fermoy Lasses}

More tunes I've known forever. I put this set together for the novelty of having three types of tunes (a slide, a jig, and a reel) that are all very similar but increasing in "notes per measure".

\subsection{The Road to Lisdoonvarna / The Swallowtail Jig / The Kesh Jig}

This started out as the previous set, but I was preparing to play a St. Patrick's Day gig with Chops MacConnie. I presented this set. He didn't know The Fermoy Lasses, and suggested putting The Kesh in its place. This has since become a regular set at the West Lafayette session.

\subsection{The Frost is All Over / Kitty Lie Over / Cunla}

All three of these are jigs that are songs as well. If I'm remembering correctly I learned all of them from the singing of Johnny Moynihan. What I'm calling The Frost is All Over and Kitty Lie Over are two different melodies that the same lyrics are sung to. Famously Johnny Moynihan sang these together with Planxty. 

The song Cunla, as I learned several years later, is sung to the first three parts of the five part jig The Freize Britches.

\subsection{Where I Told Her I Loved Her and Sprained Her Ankle / Gander in the Pratie Hole / The Persistence of Noel Reid}

The first and third of these are my own compositions. The first I named for the place where I first said "I love you" to the woman I would later marry. It was a pier made of granite blocks in Stonington Point in Connecticut. On the walk back in to firm ground she slipped and sprained her ankle. 

The Persistence of Noel Reid is named for the night I was introduced to Irish set dancing and the tunes they are danced to. I was living in New Orleans at the time. This would be about 2003. I had fallen in with a group of shanty singers, The NO Quarter Shanty Krewe. Some of them also played Irish music, and one night they invited to a ceili. I didn't know what a ceili was. I was told that there would be live music and dancing. We would get to sing a little, and people would be generally quite nice. 
I arrived where I was told go but was confused upon my arrival as I looked around and say people playing pool, country music on the juke box and a lot of leather jackets by the bar. I was pretty sure that there was only one Mick's Pub in Mid City. I was then relieved to see stairs in the corner. I went up the stairs to find a room with at least a dozen musicians lined up along two walls, just enough room for two sets of dancers, and everyone else pressed up against the walls, carrying on conversations while trying to stay out of the way. 

Confident that I was, in fact, in the right place, I went back to the bar, grabbed a drink, and re-entered the room. At this point they dance they were doing had just finished, and a man named Noel was yelling "We need more men for this set". Keep in mind that at this point I was about 25 years old. I had never danced before. I did not think that I was interested in starting to dance. But he urged me toward the floor, I thought "why not" and set down my drink. Noel placed me in front of someone, said "This is Amy. She knows what she's doing." And walked away. 

I ended up having a blast the rest of the night. And ever since I have been hooked on Irish traditional music, and set dancing. I attended the weekly set dancing lesson, the monthly ceilis, and the regular sessions for that last year that I lived in New Orleans. 

Gander in the Pratie Hole I first tried to learn from a book I was given. I think I actually learned it later from a Planxty album. 

\subsection{Where I Told Her I Loved Her and Sprained Her Ankle / Coffee on the Bricks / The Persistence of Noel Reid}

This set is simply the previous set, except that I replaced Gander in the Pratie Hole with Coffee on the Bricks, which is another tune of my own composition. This would make for the first time I was able to put together a set of three tunes all of which were mine. 

Coffee on the Bricks is named for both the brick paved historic Main St in St Charles, Missouri, and Picasso's Coffee House on that street. 

During the time we lived in St Charles, I was staying home with the kids. We only lived a couple of blocks from Main St so I frequently put the kids in a stroller and walked them down to Main St, and Frontier Park on the Missouri River. This also usually involved a stop at Picasso's, where much of my social life revolved around at the time. 

As the kids got a little older we would make the walk down to Main St without the stroller, and I would often buy the kids hot chocolate as well. My youngest, Ren, would refer these walks as "getting coffee on the bricks".

\subsection{The Walls of Liscarroll / Behind the Haystack / Merrily Kissed the Quaker}

At the moment I'm having trouble remembering where I learned each of these specific tunes. I know that I put them together because I thought they worked well together and made nice transitions from one to the next. 

\subsection{Tar Road to Sligo / The Cliffs of Moher / Salt River Road}

The first two I learned as set from one of John Whelan's albums. Salt River Road is my composition. I named it for a road in St Charles, Missouri. I remember it as being one of the first exit signs I would see on the highway after leaving our house. 

\subsection{Maggie in the Woods / Peggy Lettermore}

These I learned from the regular Wednesday night sessions at Riley's Pub in St Louis. I believe Jerry Maloney put these together as a set. 

\subsection{Lilting Banshee / Coffee / Tripping up the Stairs}

I'm not sure where specifically I learned Lilting Banshee, or Tripping up the Stairs. I know that they were in my repertoire pretty early on. Coffee I know was popular among the contra dance players I was around a lot in Connecticut. And I think we all learned it from one of The Portland Collection books. \footnote{I need to look up who the composer is}

\subsection{Dinky's / The Ash Plant / Return to Milltown}

I don't remember where I learned Dinky's. It might have been from hearing it played by The Great Bear Trio, and then looking it up later. The Ash Plant I learned from the Noel Hill and Tony McMahon album I gCnoc na Graí. Return to Milltown I learned from a Greenfield Dance Band album. \footnote{need to look up composer}

\subsection{The Worn Petticoat / Tehan's Favorite / Eileen O'Riordan's}

All three of these tunes I learned from Matt Cunningham's boxed set of CDs of music recorded for set dancing. They were not put together as a set, but they were all recorded for figures of the Ballyvouney Jig Set. 

\subsection{The Oddfellows in Plainville / Lydia Has no Faith in Cats / The Kesh Jig}

The first two are my compositions. I put the Kesh at the end in order to fill it out as a three tune set.

The naming of The Oddfellows in Plainville... About 2004 I was living in an apartment in Plainville, Connecticut. The view from my living room window on the second floor was of an Oddfellow's Hall across the street. At some point the words Oddfellows in Plainville crossed my mind and I was tickled. 

Lydia Has no Faith in Cats is named for a poem that was in a book called American Murder Ballads. \footnote{Look up author and such} Lydia was arrested in my hometown of Derby, Connecticut after she had murdered several husbands by poisoning them with rat poison(cyanide). As a way of explaining why Lydia was buying so much rat poison, the poem started:
\begin{center}
    "Lydia Sherman was plagued with rats.\\
    Lydia had no faith in cats" \\
\end{center}

\subsection{Edna's Vase / Hillgrove's Waltz}

This is a pair of waltzes of my own composition. Edna is my maternal grandmother's name. I named this tune for her sense of humor. She was diabetic, which was the cause for her having a leg amputated at the knee. A thing that happens when one is given a prosthetic limb, eventually the swelling of the stump lessens and they will have to be fitted for a new prosthetic. After she received her second prosthetic, the first one continued to live in the dining room with flowers in it. 

Thomas Hillgrove was a dance instructor in New York in the 19th\footnote{Check that} century who wrote a book of dances.\footnote{Find that book info} In the introduction was this quote.\footnote{Look up that quote.}

\subsection{The Banshee / The Merry Blacksmith / The Hunter's Purse}

I don't remember where I learned The Banshee. I feel like I learned it from the Riley's session in St Louis. I think The Merry Blackthorn might be the first reel that I learned. The Hunter's Purse I only learned recently from Chuck Whittmore at the West Lafayette session. I had put the first two together due to the similarity in their A parts. Chuck had played The Merry Blacksmith with The Hunter's Purse for a long time, so I stuck it at the end of this set.

\subsection{Joe Cooley's / Toss the Feathers / Drowsey Maggie}

Like most people, I learned Drowsey Maggie early in my playing. Toss the Feathers I learned from a couple different recordings. I do not remember when and where I learned Joe Cooley's. 

\subsection{The Lark in the Morning}

The Lark in the Morning I very much remember learning from the the Riley's session in St Louis.

\subsection{Bohola Jig / The Blackthorn Stick / Calliope House}

All three of these I learned from the Riley's session in St Louis. They played Bohola in a different set. The Blackthorn Stick, which they called Coach Road to Sligo, they followed with Calliope House.

\subsection{The Maid Behind the Bar / The Musical Priest / The Silver Spear}

I'm sure I learned the first two at sessions, but I'm not remembering where and when. The Silver Spear, I learned back at sessions in Connecticut. The one was popular in both the Irish music and contra dance circles I was running in. 

\subsection{Oak Cliff Road / O'Dowd's Pitch / Kitty on the Rail}

These are all my own compositions. Oak Cliff Road is named for the cemetary, Oak Cliff Cemetary, that was across the street from the house I grew up in. I spent a lot of time in that cemetary. It is where I learned to ride a bike. It is where I learned to play catch. Many photos were taken there while I was in art school. 

O'Dowd's Pitch is named for something that was said in an episode of the podcast Skeptoid. The episode was about the Shumann resonance\footnote{spelling}. Don't worry about what that is, it's some pseudoscientific nonsense. But the host, Brian Dunning, was talking about someone named O'Dowd selling something based on said nonsense and Brian said the words "O'Dowd's pitch was..." And that just sounded like tune name to me. 

Kitty on the Rail... I wrote this one while we were living in Pennsylvania. The name came from one Sunday morning while we were on our way to the Quaker meeting at Swathmore College and we drove past a house where I saw a cat sitting on a handrail on the front steps. I was struck by the sight of this rather round cat sitting on this narrow railing with its feet under its body, and its body essentially wrapped around this railing beneath it.

\subsection{The Whistling Wrangler / Face the Table / Untitled Polka}

These are three polkas of my own composition. The Whistling Wrangler I named for the Jeep Wrangler that my wife used to own. It had a tiny leak in the exhaust which made a whistling sound that sputtered in time with the cylinders firing. This kind of made it sound like a flying car from The Jetsons. Face the Table is something that I found myself saying very frequently when the kids were little. Unsurprisingly, most things were more likely to draw their attention away from their plate. Untitled should be self explanitory.

\subsection{The Fat Cardinal / Untitled reel}

I don't remember when I wrote these. I know that they were written fairly close together in time. The Fat Cardinal is named for an unusually large and round red bird I had noticed while looking out a window into the backyard.

\subsection{Fred Finn's / Sailing into Walpole's Marsh}

This set was taken from the album Andy Irvine and Paul Brady. I learned it when I first moved to West Lafayette and started getting together to play with Robert Freeman and Cliff Harrison. Learning this set was the suggestion of Cliff. 

\subsection{The Drunken Landlady / Colonel Rodger's Favorite / The Happy Days of Youth}

This set comes from Liam O'Flynn's album The Piper's Call. This set was also learned in that time when Robert, Cliff and I would get together to play. This was also at Cliff's suggestion.

\subsection{Miss Monaghan's / The Earl's Chair / Bird in the Bush}

I learned this set from a recording I made while at a session I attended while visiting St Louis. The sesion was at Tig\'in. I believe Tim Yau, John Bouldwan, and (that accordion play) were in attendance.

\subsection{Down The Hill}

I learned this from the album Traditional Music of Ireland by James Kelly, Paddy O'Brian, and D\'aith\'i Sproule.

\subsection{An Phis Phliuch / The Butterfly / The Kid on the Mountain}

This set was brought to the West Lafayette session by piper Taylor Stirm.

\subsection{Fisher's Hornpipe / Staten Island Hornpipe / St Anne's Reel}

These are all tunes that I learned early in my playing. They are all popular among the contra dance players in Connecticut. These three as a set were brought to the West Lafayette session by fiddler Donald Jones. 

\subsection{Ballydesmond Polka \#2 / Ballydesmond Polka \#3 / Julia Clifford's} 

This set was brought to the West Lafayette session by Dr. David Minton.

\subsection{Drops of Brandy / Hardiman the Fiddler / A Fig for a Kiss} 

The first two I learned from the Riley's session in St Louis. A Fig for a Kiss I learned from Paul and Susan Heasty at the West Lafayette session.

\subsection{Tom McCann's / Tom McCann's / John Egan's} 

These are not the names of the first two. This set was regurlarly played by Tom McCann, an accordion player I would see at sessions around Philadelphia. 

\subsection{The Golden Gardens / The Salmon's Leap} 

These are both written by Randal Bays. I learned them from flute player Dr. Lisa Gilbert in St Louis.

\subsection{John Brennan's / Father Kelly's / Father Tim's} 

John Brennan's and Father Kelly's were brought to the West Lafayette session by Randall Hauk. Father Tim's was written by Elisha Hickey.

\subsection{Bill Sullivan's / Britches Full of Stitches} 

This pair is a classic Jackie Daly set.

\subsection{The Pipe on the Hob \#1 / The Pipe on the Hob \#2} 

These two were brought to the West Lafayette session by Randall Hauk.

\subsection{Tatter Jack Walsh / Jimmy Ward's} 

I learned this set from Jim and Kate Smith at the Golden Ace session in Indianapolis.

\subsection{The Concertina Reel / The Mountain Road / The Dunmore Lasses} 

This set I learned from Lew Truex from Indianapolis.

\subsection{Tatter Jack Walsh / Tobin's Favorite / Mooncoin} 

This set was brought to the West Lafayette session by Taylor Stirm.

\subsection{Tobin's Favorite / Out on the Ocean / Tripping up the Stairs} 

Another set brought to the West Lafayette session by Randall Hauk.

\subsection{The Bank of Turf / Saddle the Pony / My Darling Asleep} 

Another set brought to the West Lafayette session by Randall Hauk.

\subsection{Donnybrook Fair / Old Had You Have Killed Me / Haste to the Wedding} 

This set was brought to the West Lafayette session by Paul Heasty.

\subsection{Fifty Cent Piece / Three Little Drummers / When Sick is it Tea You Want?} 

Another set was brought to the West Lafayette session by Paul Heasty.

\subsection{Rakish Paddy / The Old Bush} 

This set was brought to the West Lafayette session by Taylor Stirm.

\subsection{Cheif O'Neill's Favorite / The Belfast Hornpipe} 

I think that it was Dr. David Minton that brought Cheif O'Neill's to the West Lafayette session. And the Belfast Hornpipe was brought by Chuck Whittmore.

\subsection{Reel de Montebello / Evit Gabriel / Reel de Montreal}

These are some of the small handfull of Quebecois tunes that I know, so at some point I decided to put them together as a set. 

\subsection{The Freize Britches}

This one came to the West Lafayette session from Taylor Stirm.

\subsection{Lucy Farr's / Bill Malley's / Kilnamona}

Lucy Farr's I learned at the Riley's session in St Louis. Chuck Whittmore brought Bill Malley's to the West Lafayette session, and I found Kilnamona played by Martin Hayes when looking for something to go with Bill Malley's.

\subsection{The Church St Polka / The Happy Polka}

I heard this set played by Conal \'O Gr\'ada on his album and felt compelled to learn it. 

\subsection{Come Back Paddy Reily / Waltz I Learned from John Winston}

I learned these from John Winston.

\subsection{The Humors of Ballylaughlin}

I believe I learned this one from Chuck Whittmore. My playing of it was refined while taking lessons from Se\'an Gavin.

\subsection{Donnybrook Fair / The Rambling Pitchfork / The Kilashandra Lasses}

The Rambling Pitchfork I learned from Conal \'O Gr\'ada's album. That same recording is where the idea to put Donnybrook Fair with Rambling Pitchfork. The Kilashandra Lasses I learned from the Bua album Down the Green Field.

\subsection{Sonny's Mazurka}

I learned this early on back at the sessions in Connecticut. I had to go re-learn it when mazurkas had come in conversation at the West Lafayette session.

\subsection{Mickey Dalton's \#3 / The Kerry Polka / Sweeney's Polka}

This set I pulled from one of Shannon Heaton's Virtual guided sessions, although I had already known The Kerry Polka and Sweeney's. 

\subsection{Young Tom Ennis / The Rakes of Kildare / Jimmy Ward's}

Young Tom Ennis and The Rakes of Kildare I learned from my lessons with Se\'an Gavin. Jimmy Ward's I learned from Jim and Kate Smith at the Golden Ace sesion in Indianapolis.

\subsection{The Foxhunter's Jig}

I think I might have learned The Foxhunter's Jig from someone on TikTok? Or at least was reminded of it, and then looked up some recordings of it.

\subsection{The Yellow Cow / O'Connel's Trip to Parliment}

These I learned during my lessons with Se\'an Gavin.

\subsection{O'Mahoney's / The Stack of Barley / Bantry Bay}

All three of these I learned from Se\'an Gavin. I put them into this set as Julia Clifford and Denis Murphy played O'Mahoney's and The Stack of Barley together. And Michael Coleman plaed The Stack of Barley and Bantry Bay together. Although that recording is only named The Stack of Barley.

\subsection{The Bucks of Oranmore}

Another tune I learned from Se\'an Gavin.

\subsection{The Torn Jacket / Devaney's Goat / Come West Along the Road}

I also learned all three of these from Se\'an Gavin. I put them together as a set myself.

\subsection{Cronin's Hornpipe / The Little Stack of Wheat / Johnny Will You Marry Me?}

More tunes I learned from Se\'an Gavin, and put together myself.

\subsection{Catharsis / The Wizard's Walk}

Both of these are popular contra dance tunes. The Wizard's Walk has a dance written for it. Catharsis came up when Elisha Hickey and I were the only two who showed up for a session at a church fish fry. Eli had recently learned Catharsis, and played it. Afterward I was thinking about what should go with it. It occoured to me that The Wizard's Walk would likely go nicely. So I showed it to Eli and we both went and learned it. 

\subsection{The Humors of Glendart / T\'a an Coileach Ag F\'ogairt an Lae / The Mouse in the Mug}

The Humors of Glendart and T\'a an Coileach Ag F\'ogart an Lae I learned from Se\'an Gavin. The Mouse in the Mug I learned from Kevin Crawford's album In Good Company. 

\subsection{Ballydesmond Polka \#1 / Ballydesmond Polka \#2 / Ballydesmond Polka \#3}

I had learned these at different times, but only more recently had I listened to The Star Above the Garter album from Julia Clifford and Denis Murphy so I felt compelled to re-learn them.

\subsection{The Star Above the Garter / The Dingle Regatta}

I'm not sure where I learned Star Above the Garter. I think it's one that I just heard enough times in enough places that it came out of my fingers. I do remember having to ask other people if anyone knew its name. 

The Dingle Regatta I learned early on at the session at City Steam in Hartford Connecticut. That session was hosted by P.V. O'Donnel, Eddie Burke, Tom Walsh, and John Kalinowski. This was the tune that ended the set that they always ended the session with.

\subsection{The Blackthorn Stick / The Coach Road to Sligo}

This set I learned from Chuck Whittmore. He calls both of these tunes The Blackthorn Stick. The Coach Road to Sligo I learned at the Riley's session in St Louis. 

\subsection{Willie Coleman's / The Old Favorite}

The  both of these are tunes that I've heard forever, but only recently went and learned then properly, and with the intention of putting them together.

\subsection{O'Carolan's Welcome}

This one got brought to the West Lafayette session by Robert Freeman. After hearing Robert play it a few times I went to the Chieftans 2 album to fill in the gaps. 

\end{document}

